% !TEX TS-program = pdflatex
% !TEX encoding = UTF-8 Unicode

% This is a simple template for a LaTeX document using the "article" class.
% See "book", "report", "letter" for other types of document.

\documentclass[11pt]{article} % use larger type; default would be 10pt

\usepackage[utf8]{inputenc} % set input encoding (not needed with XeLaTeX)

%%% Examples of Article customizations
% These packages are optional, depending whether you want the features they provide.
% See the LaTeX Companion or other references for full information.

%%% PAGE DIMENSIONS
\usepackage{geometry} % to change the page dimensions
\geometry{a4paper} % or letterpaper (US) or a5paper or....
% \geometry{margin=2in} % for example, change the margins to 2 inches all round
% \geometry{landscape} % set up the page for landscape
%   read geometry.pdf for detailed page layout information

\usepackage{graphicx} % support the \includegraphics command and options

% \usepackage[parfill]{parskip} % Activate to begin paragraphs with an empty line rather than an indent

%%% PACKAGES
\usepackage{booktabs} % for much better looking tables
\usepackage{array} % for better arrays (eg matrices) in maths
\usepackage{paralist} % very flexible & customisable lists (eg. enumerate/itemize, etc.)
\usepackage{verbatim} % adds environment for commenting out blocks of text & for better verbatim
\usepackage{subfig} % make it possible to include more than one captioned figure/table in a single float
% These packages are all incorporated in the memoir class to one degree or another...

\usepackage{scrextend}
\usepackage{blindtext}
\addtokomafont{labelinglabel}{\sffamily}
\usepackage{graphicx}
\usepackage{fancyvrb}
\newcommand{\verbatimfont}[1]{\renewcommand{\verbatim@font}{\ttfamily#1}}

%%% HEADERS & FOOTERS
\usepackage{fancyhdr} % This should be set AFTER setting up the page geometry
\pagestyle{fancy} % options: empty , plain , fancy
\renewcommand{\headrulewidth}{0pt} % customise the layout...
\lhead{}\chead{}\rhead{}
\lfoot{}\cfoot{\thepage}\rfoot{}

%%% SECTION TITLE APPEARANCE
\usepackage{sectsty}
\allsectionsfont{\sffamily\mdseries\upshape} % (See the fntguide.pdf for font help)
% (This matches ConTeXt defaults)

%%% ToC (table of contents) APPEARANCE
\usepackage[nottoc,notlof,notlot]{tocbibind} % Put the bibliography in the ToC
\usepackage[titles,subfigure]{tocloft} % Alter the style of the Table of Contents
\renewcommand{\cftsecfont}{\rmfamily\mdseries\upshape}
\renewcommand{\cftsecpagefont}{\rmfamily\mdseries\upshape} % No bold!

%%% END Article customizations

%%% The "real" document content comes below...

\title{Project P7: Design an A/B Test}
\author{Peter Eisenschmidt}
%\date{} % Activate to display a given date or no date (if empty),
         % otherwise the current date is printed 

\begin{document}
\maketitle

\section{Experiment Design}

\subsection{Metric Choice}

The following metrics have been selected as \textbf{Invariant Metrics}:
\begin{labeling}{Click-through-probability}
\item [Number of cookies] The number of unique cookies to visit the page should not be affected by the experiment, as someone visiting the course overview page has not seen the changes yet. 
\item [Number of clicks] The same applies to the number of clicks on the "Start Free Trial" button; there should not be any impact of the experiment on this metric
\item [Click-through-probability] As the CTR is defined as the number of unique cookies to click the "Start free trial" button divided by the number of unique cookies to view the course overview page (both of which are invariant metrics), the click-through-probability is also an invariant metric
\end{labeling}
\noindent These three metrics are not impacted by the experiment and hence one can expect similar distributions between control and experiment groups.
\medskip

\noindent The following metrics have been selected as \textbf{Evaluation Metrics}:
\begin{labeling}{Gross conversion}
\item [Gross conversion] Being defined as the number of user-ids to complete checkout and enroll in the free trial divided by the number of unique cookies to click the "Start free trial" button, one would expect a lower gross conversion for the experiment as for the control group. The goal of the tested change is to reduce the number of frustrated students, so you could expect that students that are likely to drop out with the current design are filtered out early and do not complete the checkout.
\item [Retention] Similarly, you would expect an increased retention as a result of the experiment, as the number of students that complete the checkout should reduce. At the same time, the number of students to make at least one payment should remain the same.
\item [Net conversion] Net conversion is the combination of the two previously mentioned metrics. It is expected that net conversion remains the same for both control and experiment group, as the number of students to remain enrolled past the 14-day boundary as well as the number of unique cookies to click the "Start Free Trial" Button should remain the same.
\end{labeling}
\noindent For each of these metrics, a practical significance boundary $d_{min}$ is defined. This indicates the minimum difference that needs to be observed between control and experiment group in order to determine whether the change is meaningful or not. This is important for the decision to whether or not launch the change.

For the above given evaluations metrics, the practical significance boundaries are $d_{min} = .01$ (for gross conversion and retention) and $d_{min} = .0075$ (for net conversion)

\subsection{Measuring Standard Deviation}

The analytical estimate of the standard deviation can be calculated as follows:
\begin{equation}
     \sigma = \sqrt{\frac{p(1-p)}{N}}
\end{equation}
\noindent where the probabilities are given in the baseline values:
\begin{itemize}
\item Probability of enrolling, given click (Gross Conversion): $p= .20625$
\item Probability of payment, given enroll (Retention): $p=.5300$
\item Probability of payment, given click (Net Conversion): $p=0.1093125$
\end{itemize}

\noindent Given that the sample size to visit the course overview page is 5000 cookies, the number of units of analysis for each metric can be calculated as follows. For gross conversion, it is given by:
\begin{equation}
     N = \frac{PageViews \times Cookies_{ClickFreeTrial}}{Cookies_{ViewPagePerDay}} = \frac{5000 \times 3200}{40000} = 400
\end{equation}
For retention it can be calculated as:
\begin{equation}
     N = \frac{PageViews \times Enrollments}{Cookies_{ViewPagePerDay}} = \frac{5000 \times 660}{40000} = 82.5
\end{equation}
For net conversion, it is the same as gross conversion:
\begin{equation}
     N = \frac{PageViews \times Cookies_{ClickFreeTrial}}{Cookies_{ViewPagePerDay}} = \frac{5000 \times 3200}{40000} = 400
\end{equation}
This results in the following standard deviations:
\begin{itemize}
\item Gross Conversion: $\sigma = .0202$
\item Retention: $\sigma = .0549$
\item Net Conversion: $\sigma =  .0156$
\end{itemize}


\subsection{Sizing}
Number of Samples vs. Power\medskip

\noindent Duration vs. Exposure

\section{Experiment Analysis}
\subsection{Sanity Checks}

\subsection{Result Analysis}

\subsection{Recommendation}

\section{Follow-Up Experiment}

\end{document}